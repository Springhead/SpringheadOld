% 1.4.Solution.tex
%	Last update: 2019/07/10 F.Kanehori
\newpage
\subsection{対処法}
\label{subsec:Solution}

\noindent
前節(\ref{subsec:Problems})で示した問題点への対処法について述べます。

\bigskip
\noindent
\bf{ソースファイルの整合性}
\begin{narrow}[20pt]
	Springhead Libraryのソースツリーにあるプロジェクトディレクトリ
	(\tt{Base}, \tt{Collision}, $\ldots$)を
	直接\tt{add\_subdirectory}すれば問題ありません。
\end{narrow}

\medskip
\noindent
\bf{ビルドの最適性(無駄なコンパイル)}
\begin{narrow}[20pt]
	Springhead Library, \it{App1, App2\,}などで
	オブジェクトが生成される場所を共通化してしまうことで
	この問題を回避することとします。
	具体的には、Springhead Libraryソースツリーの中にオブジェクトの共通格納場所を作り、
	Springhead Libraryおよびすべてのアプリケーションでオブジェクト格納領域が
	そこを指すようにlinkを張ることとします。

	\medskip
    	\begin{narrow}\begin{minipage}{\textwidth}
		\dirtree{%
			.1 \hspace{-10mm}.../Springhead/core/src/ \Anno{ソースツリー}.
			.2 Base/ \Anno{(*s1)}.
			.3 <\it{platform\,}>/.
			.4 <\it{VS-Version\,}>.
			.5 /Base.dir/ \Anno{オブジェクト共通格納領域(*o1)}.
			.3 :.
			.2 :.
			.2 \build \Anno{ビルドツリー (他の場所でも構わない)}.
			.3 Base/.
			.4 Base.dir/ \Anno{(*o1)にlinkを張る}.
			.2 :.
		}
		\medskip
    	\end{minipage}\end{narrow}
    	\begin{narrow}\begin{minipage}{\textwidth}
		\dirtree{%
			.1 \hspace{-10mm}.../Application/ \Anno{ソースツリー}.
			.2 Base/ \Anno{(*s1)を指すようにする}.
			.2 :.
			.2 \build \Anno{ビルドツリー (他の場所でも構わない)}.
			.3 Base/.
			.4 Base.dir/ \Anno{(*o1)にlinkを張る}.
			.3 :.
		}
		\medskip
  	\end{minipage}\end{narrow}
	\begin{center}図\ref{subsec:Solution}.1 ビルドの最適性への対処法\end{center}
	このlinkを張る作業は、アプリケーションの\cmake 時に行なうものとします。
\end{narrow}

\medskip
\noindent
\bf{プロジェクトファイルの整合性}
\begin{narrow}[20pt]
	ビルドの最適性の場合と同様、
	プロジェクトファイルも共通化することでこの問題に対処します。
	具体的には、各プロジェクトファイルは
	Springehad Libraryビルドツリーにあるものを基準とし、
	すべてのアプリケーションのプロジェクトファイルは、そこへのlinkします。

	\medskip
    	\begin{narrow}\begin{minipage}{\textwidth}
		\dirtree{%
			.1 \hspace{-10mm}.../Springhead/core/src/ \Anno{ソースツリー}.
			.2 Base/ \Anno{(*s1)}.
			.2 :.
			.2 \build \Anno{ビルドツリー (他の場所でも構わない)}.
			.3 Base/.
			.4 Base.vcxproj \Anno{(*p1)}.
			.2 :.
		}
		\medskip
    	\end{minipage}\end{narrow}
    	\begin{narrow}\begin{minipage}{\textwidth}
		\dirtree{%
			.1 \hspace{-10mm}.../Application/ \Anno{ソースツリー}.
			.2 Base/ \Anno{(*s1)を指すようにする}.
			.2 :.
			.2 \build \Anno{ビルドツリー (他の場所でも構わない)}.
			.3 Base/.
			.4 Base.vcxprj/ \Anno{(*p1)へのlinkとする}.
			.3 :.
		}
		\medskip
  	\end{minipage}\end{narrow}
	\begin{center}図\ref{subsec:Solution}.2プロジェクトファイルの最適性への対処法
	\end{center}

	\medskip
	ただしこれでは不完全で、\App{1}で実施したプロジェクトファイルへの変更が
	\App{2}に伝わりません。
	このため\App{1}でプロジェクトファイルを変更した場合には、
	その変更をSpringhead Libraryビルドツリーにあるプロジェクトファイルに
	反映させるものとします。
	つまり、Springhad Libraryのビルドツリーにあるプロジェクトファイルを
	常に最新の状態に保つということです。

	この作業はアプリケーションのビルド時に行なうものとし、
	そのために特別なターゲット\tt{sync}を作成して、
	このターゲットが毎回ビルドに先立って実行されるようにします。
\end{narrow}

% end: 1.4.Solution.tex
