% 1.3.Problems.tex
%	Last update: 2019/10/03 F.Kanehori
%\newpage
\subsection{CMakeを使用した場合の問題点}
\label{subsec:Problems}

\def\App#1{\it{App#1\,}}
\def\App#1{\it{App#1\,}}

\noindent
前節の図\ref{fig:ApplicationTree}に示したような
アプリケーション\App{1}の他にアプリケーション\App{2}があったとして、
その両方を同時に開いて作業を行なう場合を考えます。
ここでは\App{1}と\App{2}が共通して参照する\SprLib のプロジェクトについてのみ考えます。

\bigskip
\noindent
\bf{ソースファイルの整合性}
\begin{narrow}[20pt]
	ソースファイルの整合性には問題がありません。
	\App{1}と\App{2}のどちらも\SprLib のソースツリーを指すようにしてありますから、
	どちらで実施したソースファイルの変更も直ちに他方に反映されることになります。
	(同じファイルを参照しているのですから当然です)

	 ソースファイルの追加や削除を実施したとしても
	ソースファイルの整合性自体が崩れるわけではありません。
	ただし、この場合には\KQuote{\bf{プロジェクトファイルの整合性}}の問題が発生します。
\end{narrow}

\medskip
\noindent
\bf{ビルドの最適性(無駄なコンパイル)}
\begin{narrow}[20pt]
	\App{1}と\App{2}のビルドツリーは異なるものですから、
	\App{1}でソースを変更してビルドしたとしても
	そのビルド結果(オブジェクトファイル)がそのまま\App{2}に反映されることはありません。
	これらは異なるファイルです。

	 \App{2}でソースの変更を反映させようとすれば
	\App{2}で再ビルドを行なう必要があります (この時点で変更されている
	ソースファイルが再コンパイルされ、オブジェクトファイルの整合性がとれます)。
	従来はオブジェクトファイルも共有されていましたから、
	このような無駄なコンパイルは発生しませんでした。
\end{narrow}

\medskip
\noindent
\bf{プロジェクトファイルの整合性}
\begin{narrow}[20pt]
	ソースファイルの内容を修正しただけならば
	プロジェクトファイルが変更されることはありません。
	しかし、ソースファイルの追加や削除を行なったならば
	プロジェクトファイルにも変更が及びます
	(Visual Studio上でソースの追加・削除を行なってセーブを行なった場合、
	もしくは\App{1}での変更後に再\cmake を行なった場合)。

	 プロジェクトファイルもオブジェクトファイルと同様に
	ビルドツリー内に生成されますから、
	\App{1}で実施した変更が\App{2}に反映されることはありません。
	しかも\App{2}で再ビルドしたとしても\App{1}での変更が反映されることはありません。
	\App{2}のプロジェクトファイルは\App{1}のものとは異なるファイルですから
	\App{2}は依然として古い状態を残したままなのです。

	\indent
	 これを解消するには\App{2}で再度\cmake を実行する必要がありますが、
	問題は\KQuote{\bf{いつ再\cmake が必要なのかが分からない}}ということです。
\end{narrow}

% end: 1.3.Problems.tex
