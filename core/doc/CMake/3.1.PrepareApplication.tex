% 3.1.PrepareApplication.tex
%	Last update: 2019/07/17 F.Kanehori
%\newpage
\subsection{準備}
\label{subsec:PrepareApplication}

\noindent
アプリケーションと並行してSpringheadライブラリを開発する場合には、
\UpKQs \ref{subsec:Problems} CMakeを使用した場合の問題点\UpKQe
で示した問題に対処する必要があるため、
ここで述べる方法に従って作業を進めてください
(\KQuote{\ref{subsec:Solution} 対処法}で示した方策が組み込まれます)。

\medskip
以下、Springheadライブラリをダウンロードしたディレクトリを\SprTop{} 、
アプリケーションプログラムを作成するディレクトリを\AppTop{}として説明します。

\bigskip
\noindent
\AppTop{}に移動してください。

\bigskip
\noindent
配布されたファイル\CMakeTopdir{.dist}を\CMakeTopdir{}という名前で、
\CMakeLists{.Dev.dist}を\CMakeLists{}という名前でコピーします
(誤コミットを防止するためにもリネームではなくコピーしてください)。

\medskip
\begin{narrow}[s][15pt]
	\CmndBox{%
		> chdir C:/Develop/Application\\
		> copy C:/Springhead/core/src/CMakeTopdir.txt.dist CMakeTopdir.txt\\
		> copy C:/Springhead/core/src/CMakeLists.txt.Dev.dist CMakeLists.txt
	}
\end{narrow}

\bigskip
\begin{description}
    \item[\tt{CMakeTopdir.txt}の編集] \\
	\CMakeTopdir{}にSpringheadライブラリをダウンロードしたディレクトリを設定します。
	これは、CMakeにSpringheadのソースツリーの場所を教えるために必要な設定です。

	\begin{narrow}[s][15pt]
		\CmndBox{%
			\#set(TOPDIR "C:/Springhead")\\
			\hspace{10pt}$\downarrow$\\
			set(TOPDIR \SprTop{})
		}
	\end{narrow}
	\Vskip{.5\baselineskip}

    \item[\tt{CMakeLists}の編集] \\
	\begin{enumerate}
	    \item
		プロジェクト名を設定します(11行目)。

		\begin{narrow}[s][5pt]
			\CmndBox{%
				set(ProjectName "Project")\\
				\hspace{10pt}$\downarrow$\\
				set(ProjectName "<\it{MyApplication}>")
			}
		\end{narrow}
		\Vskip{.5\baselineskip}

	    \item
		Customization section (19行目以降)の変更。\\
		各変数の意味は次のとおりです。

		\begin{narrow}[2pt]
		\begin{tabular}{|l|l|}\hline
		    \tt{OOS_BLD_DIR} &
			CMakeの作業領域(ディレクトリ)の名前\\
			& (本ドキュメントで\build としているもの)。\\\hline
		    \multicolumn{2}{|l|}{
			\tt{CMAKE_CONFIGURATION_TYPES}} \\
			& ビルド構成。\\\hline
		    \tt{EXCLUDE_SRCS} &
			ビルドの対象から外すファイル。\\
			& 直前で指定しているとおり、
			ビルド対象のデフォルト\\
			& は\tt{*.cpp *.h}です。\\\hline
		    \tt{SPR_PROJS} &
			アプリケーションに組み込むSpringheadライブラリ\\
			& のプロジェクト名。\\
			& 不要なプロジェクト名を削除します。
			この中にRun-\\
			& Swigを含めてはいけません。\\\hline
		    \tt{ADDITIONAL_INCDIR} &
			追加のインクルードパス指定。\\
			& 現在のディレクトリは\tt{\$\{CMAKE_SOURCE_DIR\}}で参照\\
			& できます。
			\\\hline
		    \tt{ADDITIONAL_LIBDIR} &
			追加のライブラリパス指定。\\\hline
		    \tt{ADDITIONAL_LIBS} &
			追加のライブラリファイル名。\\\hline
		    \tt{EXCLUDE_LIBS} &
			linkの対象から外すライブラリファイル名。\\
			& デフォルトで組み込まれてしまうライブラリファイル\\
			& を排除するために指定します。\\\hline
		\end{tabular}
		\end{narrow}
	\end{enumerate}
\end{description}


\bigskip
\noindent
自前でインストールしているパッケージ
(\tt{boost}, \tt{glew}, \tt{freeglut}, \tt{glui}を使用する場合には
さらに次の準備が必要となります。

配布されたファイル\CMakeConf{.dist}を\CMakeConf{}という名前でコピーして
必要な編集をします。
\KQuote{\ref{subsec:PrepareLibrary} 準備}を参照してください。

\medskip
\noindent
ビルドの条件(compileやlinkのパラメータ)を変更したいときは、
配布されたファイル\CMakeOpts{.dist}を\CMakeOpts{}という名前でコピーして
適宜変更してください。

\medskip
\noindent
以上で準備作業は終了です。

% end: 3.1.PrepareApplication.tex
