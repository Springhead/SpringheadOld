% 2.0.Application.tex
%	Last update: 2020/02/13 F.Kanehori
\newpage
\section{アプリケーションのビルド (開発者向け)}
\label{sec:Application}
\parindent=0pt

\begin{center}
\begin{tabular}{l} \hline\hline
	\makebox[.95\linewidth][l]{%
		この章での説明は、Springhead Libraryの開発者に向けたものです。
	} \\\hline\hline
\end{tabular}\end{center}

\medskip
Springhead Libraryの開発をアプリケーションの開発と同時に並行して実施する場合、
これから説明する方法を用いることをお勧めします。

\Important{%
	インストール で説明した方法で問題はありませんが、
	Visual Studio 等の開発ツールでの作業に慣れている場合には
	\bf{無駄なビルド}が気になるでしょう。
	以下に説明する方法は、これらの\bf{無駄}を少しでも無くすことを目標としています。}

\medskip
以下、\KQuote{\ref{sec:Install} インストール}に従って、
\bf{Springhead Library のインストールとビルド(正確には cmake)が
実行されていることを前提とします。}
また、\SprLib をインストールしたディレクトリを\bf{\SprTop{}}、
アプリケーションを開発するディレクトリを\bf{\AppTop{}}として説明を進めます。

\bigskip
また、\KQuote{\ref{subsec:Preparation} ビルドの準備}で示したファイルの他に、
次のファイルも使用します。

\begin{center}
\begin{tabular}{l@{\ \ ---\ \ }l}\hline
	\tt{CMakeLists.txt.Dev.dist} & アプリケーション生成用設定ファイルの雛形 \\
	\tt{CMakeSettings.txt.Dev.dist} & ビルドパラメータ変更用ファイルの雛形 \\
	\tt{CMakeTopdir.txt.dist} & ダウンロードツリー位置指定用ファイル \\\hline
\end{tabular}
\end{center}

% end: 2.0.Application.tex
