% 2.3.5.Implementation.tex
%	Last update: 2020/02/13 F.Kanehori
%newpage
\subsubsection{実装}
\label{subsubsec:Implementation}
\parindent=0pt

この節では、\KQuote{\ref{subsubsec:Solutions} 対処法}の実装に関する補足説明をします。

\medskip
\bf{ソースファイルの整合性}
\begin{narrow}
	\QCMakeLists{}に組み込みます
	(\QCMakeSettings{}の\tt{SPR\_PROJS}で定義したプロジェクトのすべてを
	\tt{add\_subdirectory}します)。\\
	\QCMakeLists{.Dev.dist}を参照のこと。
\end{narrow}

\medskip
\bf{ビルドの最適性(無駄なコンパイル)}
\begin{narrow}
	組み込まれる\SprLib の各プロジェクトに対して、
	次のことを実施します (ここではBaseを例に説明します)。

	\medskip
	\cmake (configure)実行時に作成されるオブジェクト格納ディレクトリ\\
	\hspace{10pt}\Path{C:/Develop/Application/build/Base/Base.dir}\\
	を、オブジェクト共通格納場所\\
	\hspace{10pt}\Path{C/Springhead/core/src/Base/<\it{platform}>/<\it{VS-Version}>/Base.dir}\\
	へのlink(*)にすげ替えます。

	\medskip
	オブジェクト共通格納場所は、\SprLib の\cmake (configure) 時に作成します。
	各プロジェクトの\QCMakeLists{}から
	\tt{execute\_process}で呼び出される\Path{make\_prconfig.py}を参照のこと。

	\medskip
	\Important{%
		ここで作成するlinkは、unixではsymbolic link、Windowsではjunctionです。
		これは、
		Windowsでは通常の実行権限ではsymblic linkが作成できないためです。\\
		Windowsでは、\Path{Base.dir}がjunctionなのか通常のディレクトリなのかが、
		explorerでもcommand promptでも区別がつきません。
		このことが、Q\&Aの\KQuote{\ref{subsec:QandA:CrumbleBuildOptimizeation}
		ビルドの最適性が崩れる}の原因究明を困難にする可能性を孕んでいます。
		ここでは、
		オブジェクト共通格納場所には`\tt{\_target\_body\_}'という名前の
		空ファイルを 置くことで判定の補助としています。}
\end{narrow}

\medskip
\bf{プロジェクトファイルの整合性(Visual Studio の場合)}
\begin{narrow}
	アプリケーションのソリューションファイルに新たなターゲットsyncを追加して、
	次の処理を順に実行させます。
	これにより、\SprLib 側または他のアプリケーションが実施した変更は、
	アプリケーションをビルドするだけで自動的に反映されます。

	\begin{enumerate}
	  \item	もしもアプリケーション側のプロジェクトファイルの内容と\SprLib 側の
		プロジェクトファイルの内容とが異なっていたならば、
		アプリケーション側のプロジェクトファイルを\SprLib 側にコピーする。
		これは、アプリケーション側でソースファイル構成を変更(追加・削除)を行ない、
		Visual Studio上でプロジェクトファイルを保存したとき、
		またはアプリケーション側で再度\cmake を実行した場合に相当します。

	  \item	\SprLib 側のプロジェクトファイルをアプリケーション側にコピーする。
	\end{enumerate}

	ターゲットsyncは、アプリケーションの\QCMakeLists{}で
	\tt{add\_custom\_target}として作成されます。
	また、このターゲットはアプリケーションのビルドにおいて
	必ず最初に実行されるように依存関係が設定されます。
	\QCMakeLists{.Dev.dist}および\tt{"make\_sync.py"}を参照のこと。

	\medskip
	\Important{%
		自アプリケーションで実施したソース構成の変更は、
		ビルドを実施した時点で\SprLib 側に反映されます。
		つまり、プロジェクトファイルの整合性を保つためには、
		ソース構成の変更後に少なくとも1回はビルドを実行する必要があるということです
		(syncの実行だけでもよい)。
		しかしソースを変更したならばビルドするでしょうから、
		このことが問題になることはないでしょう。}
\end{narrow}

% end: 2.3.5.Implementation.tex
